\section{Conclusions}

We used CTF as a diagnostic tool to evaluate the amount and pattern of RFS for spatial filters typically used to extract regional brain activity. We showed analytically how CTF is related to the fidelity of the extracted ROI time series and the consequences of RFS for the estimation of inter-regional connectivity (spurious and ghost interactions). These results were further validated in simulations and real data. The main caveat to keep in mind when working with the CTF is that it only shows the potential contributions of sources. In contrast, actual contributions also depend on the source amplitudes and covariances. In simulations, we showed that CTF predictions remain informative even when the default assumptions are violated, and that incorporating source covariance further increases their precision. Overall, we believe that CTF can provide valuable insights into the pattern of RFS across various contexts (ROIs, extraction pipelines, experimental conditions). To facilitate its usage, we provide an open-source implementation of the key ideas discussed in the manuscript as a Python package ROIextract. 