\section{Introduction}

% The concept of ROI

The results of neuroimaging analyses are often interpreted and communicated by referring to specific areas of the brain defined according to various parcellations, which are based on anatomical (e.g., \cite{DesikanKilliany2006, Destrieux2010}), functional (e.g., \cite{Schaefer2018}), and cytoarchitectonic (e.g., \cite{Brodmann1909}) criteria, or a combination of those (e.g., \cite{Glasser2016}). In electro- and magnetoencephalography (M/EEG), it is often convenient to define regions of interest (ROIs) and extract time courses of M/EEG activity originating from these ROIs. Such ROI-based representation reduces data dimensionality, making large-scale analyses computationally feasible. Extraction of ROI time series is widely used in basic, cognitive and clinical neuroscience for the analyses of oscillatory power \citep{DaSilvaCastanheira2021}, evoked responses \citep{Kumral2022}, functional connectivity \citep{Pellegrini2023, Brkic2023, Kapralov2024}, biophysical modeling of neural data \citep{Neymotin2020}, and real-time estimation of neural activity for brain stimulation and neurofeedback \citep{GrosseWentrup2009, Zrenner2018}.

% Problem: remaining field spread

Due to the field spread, sensor-space M/EEG recordings contain a mixture of activity from all brain sources \citep{Sarvas1987}, and inverse modeling methods are commonly used to reconstruct activity from individual brain sources. Still, the reconstruction is never perfect due to the ill-posed nature of the inverse problem \citep{SchoffelenGross2009, Palva2012}. As a result, even after source reconstruction, the extracted ROI time series also captures activity from other brain regions. We refer to this phenomenon as remaining field spread (RFS) throughout the paper.

% Consequences of the RFS for the interpretation of results

RFS may crucially affect the validity of our interpretations. Ideally, we aim to attribute the results of all analyses conducted using the extracted ROI time series (e.g., power or connectivity) to the considered ROI. However, with larger RFS, the extracted time series become less representative of the corresponding ROI, and the observed effects might originate from other regions as well. In connectivity analyses, RFS has two well-known consequences -- spurious and ghost interactions. Spurious interactions occur when the same sources contribute to the time courses of investigated ROIs. The time courses may thus appear synchronized even if there is no genuine underlying interaction \citep{Nunez1997, Nolte2004}. In contrast, ghost interactions occur when a genuine interaction between two ROIs can also be observed between other ROIs due to RFS \citep{Palva2018}. Spurious and ghost interactions pose a serious challenge to the interpretation of results, but they can be accounted for by modeling the RFS.

% The amount of RFS is unknown

The amount of RFS is generally not known. When RFS is ignored during interpretation, it is implicitly assumed that its effects are either negligible or similar across all brain areas, which may not be true. Intuitively, the amount of RFS should decrease with the distance between the locations where the signals are recorded or reconstructed. \cite{Nunez1997} showed such a decrease in sensor space, and a similar behavior can be expected for ROIs in source-space analysis. However, this idea still doesn't allow one to determine the amount of RFS more precisely for a particular analysis and interpretation. A better understanding of the RFS could be instrumental in ensuring the correct interpretation of the results.

% CTF as a diagnostic tool

The properties of linear methods for source reconstruction can be analyzed by computing their resolution matrix \citep{BackusGilbert1968, dePeraltaMenendez1996}. It is common to focus on the rows and columns of the resolution matrix, which are referred to as the cross-talk (CTF) and the point-spread function, respectively \citep{Hauk2011}. CTF shows which ground-truth sources of activity potentially contribute to the reconstructed estimate at a specific location, thereby quantifying the degree of RFS. It was used previously to compare different inverse operators \citep{dePeraltaMenendez1996, Hauk2011}, evaluate the benefit of simultaneous M/EEG recordings for source reconstruction \citep{Molins2008}, and to optimize brain parcellations \citep{Farahibozorg2018}. However, the differences between existing pipelines for extracting ROI activity and the underlying RFS have not yet been analyzed using CTF.

% Our contributions

In the present study, we apply CTF to estimate the expected amount and pattern of the RFS for different ROIs and linear extraction pipelines. We first show that every combination of linear methods for extracting ROI activity can be represented by an equivalent spatial filter and associated CTF. We then show how the expected effects of the RFS on the extraction of ROI activity and on the estimation of connectivity between ROIs can be quantified using the CTF. Using simulated data, we further investigate whether CTF can explain the differences in the extraction of activity between brain regions and extraction pipelines. Finally, we show how CTF can explain the spurious interactions observed in resting-state EEG data and ghost interactions observed during steady-state visual stimulation. To facilitate the usage of CTF, we also share the implementation of all key derivations as an open-source Python package ROIextract.
