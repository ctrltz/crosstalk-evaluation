In M/EEG analyses, it is often convenient to extract time series of activity originating from the regions of interest (ROIs). However, due to the spread of electric and magnetic fields, M/EEG recordings capture activity from all sources within the brain. Commonly used approaches for the extraction of ROI activity only partially alleviate this problem, and field spread remains a challenge even at the level of ROI time series. Because of the remaining field spread (RFS), the extracted time series captures activity not only from the considered ROI but also from other regions (not necessarily neighboring ones). The amount of RFS can strongly affect the validity of interpretations: with more RFS, the extracted time series becomes less representative of the ROI. However, neither the amount nor the pattern of RFS is usually known. In this study, we apply the cross-talk function (CTF) to analyze contributions of all sources across the brain to the extracted ROI time series, thereby quantifying the degree of RFS. With CTF, we show that the effect of RFS on the extraction of ROI activity and on the estimation of inter-regional connectivity is highly non-uniform across the cortex. In particular, ROIs farther away from the recording sensors are more likely to capture activity and connectivity from other areas. Finally, we validate this observation in simulations and complement it by investigating spurious and ghost interactions in real data. Overall, our results illustrate how CTF can be used as a diagnostic tool to quantify the effects of RFS and to evaluate pipelines for the extraction of ROI activity.
